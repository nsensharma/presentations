\documentclass [11pt]{beamer}
\usepackage{graphics}
\mode<presentation>
{
  \usetheme{metropolis}
  %\usecolortheme{dove}
%  \setbeamercovered{transparent}
}
\usepackage[autoplay,loop]{animate}
%\usepackage[orientation=landscape, size =a0, scale=2.0]{beamerposter}
\usepackage[absolute,overlay]{textpos}
\usepackage{tikz}
\usepackage[english]{babel}
\usepackage{fancybox}
\usepackage{braket}
\usepackage[utf8]{inputenc}
\usepackage{rotating}
\usepackage{enumerate}
\usepackage{blindtext}
\newcommand{\iu}{{i\mkern1mu}}
%\usepackage{amsmath}
\usepackage{multicol}
\usepackage[labelformat=empty]{caption}
\usepackage[export]{adjustbox}
\usepackage{color}

\usepackage{times}
\usepackage[T1]{fontenc}

\title{Two-phonon wobbling in $^{135}$Pr}
\date{February 13, 2018}
\author{Nirupama Sensharma}
\institute{University of Notre Dame}


\begin{document}

\maketitle

%\begin{frame}{Outline}
 %\setbeamertemplate{section in toc}[sections numbered]
 % \tableofcontents[hideallsubsections]
%\end{frame}

\section{Introduction}

\begin{frame}{A Spherical nucleus}
	\centering 
\animategraphics[height=3.1in, autoplay]{10}{images/spherical-}{0}{59}  
\\\tiny{\textit{Courtesy Dr. J. T. Matta}}
\end{frame} 

\begin{frame}
\begin{center}
\textbf{Bohr and Mottelson stated:}
\end{center} 

\textit{A common feature of systems that have rotational spectra is the existence of a 'deformation', by which is implied a feature of anisotropy that makes it possible to specify an orientation of the system as a whole.}\\
\vspace{0.6in}
\hspace{1.8in}
{\tiny {\textbf{Nuclear Structure Volume II: Nuclear Deformations (1975)}}}

\end{frame}	

\begin{frame}{Deformed Nucleus}
\begin{multicols}{2}
\begin{figure}
\includegraphics[width=0.5\textwidth]{deformed.png}
\end{figure}
\begin{center}
\begin{figure}
\includegraphics[width=0.3\textwidth]{zero.png}
\end{figure}
\Ovalbox{Rotational Band}
\end{center}
\end{multicols}
\end{frame}

\begin{frame}{Deformed Nucleus}
\begin{multicols}{2}
\begin{figure}
\includegraphics[width=0.5\textwidth]{deformed.png}
\end{figure}
\begin{center}
\begin{figure}
\includegraphics[width=0.3\textwidth]{two.png}
\end{figure}
\Ovalbox{Rotational Band}
\end{center}
\end{multicols}
\end{frame}

\begin{frame}{Deformed Nucleus}
\begin{multicols}{2}
\begin{figure}
\includegraphics[width=0.5\textwidth]{deformed.png}
\end{figure}
\begin{center}
\begin{figure}
\includegraphics[width=0.2\textwidth]{four.png}
\end{figure}
\Ovalbox{Rotational Band}
\end{center}
\end{multicols}
\end{frame}

\begin{frame}{Deformed Nucleus}
\begin{multicols}{2}
\begin{figure}
\includegraphics[width=0.5\textwidth]{deformed.png}
\end{figure}
\begin{center}
\begin{figure}
\includegraphics[width=0.2\textwidth]{six.png}
\end{figure}
\Ovalbox{Rotational Band}
\end{center}
\end{multicols}
\end{frame}

%\begin{frame}{Deformed Nucleus}
%\begin{multicols}{2}
%\begin{figure}
%\includegraphics[width=0.5\textwidth]{deformed.png}
%\end{figure}
%\begin{center}
%\begin{figure}
%\includegraphics[width=0.2\textwidth]{eight.png}
%\end{figure}
%\Ovalbox{Rotational Band}
%\end{center}
%\end{multicols}
%\end{frame}

\begin{frame}{Nuclear Shapes}
\centering
\Ovalbox{Prolate} \space \space \space \space \space \space \space \space \space \space \space \space \space \space \space \space \space \space \space \space \space \space \space \space \space \space \Ovalbox{Spherical} \space \space \space \space \space \space \space \space \space \space \space \space \space \space \space \space \space \space \space \space \space \Ovalbox{Oblate}
\begin{figure}
\includegraphics[width=\textwidth]{shapes.png}
\end{figure}
\begin{multicols}{2}
\begin{figure}
\includegraphics[width=0.25\textwidth]{triax}
\end{figure}
\Ovalbox{Triaxial Nucleus}
\begin{figure}
\includegraphics[width=0.28\textwidth]{ellipsoid.jpg}
\end{figure}
\end{multicols}
\end{frame}

\begin{frame}{Triaxial Region}
\centering
Triaxiality - A rare phenomenon!\\
\tiny {P. M\"{o}ller et. al. PRL 97, 162502 (2006)}
\begin{figure}
\includegraphics[width=0.7\textwidth]{triaxial.png}
\end{figure}
\small {Predicted regions $\rightarrow Z = 60, N = 76$ and $Z = 46, N = 66$}
\end{frame}

\begin{frame}{Triaxiality - Fingerprints (1/2)}
\vspace{-6pt}
\textbf{\large \color{red} Chirality}
\vspace{-6pt}
\begin{center}
\begin{figure}
\includegraphics[width=0.35\textwidth]{chiral.png}
\hspace{5pt}
\includegraphics[width=0.5\textwidth]{ch.png}
\end{figure}
\end{center}
\vspace{-7pt}
\begin{itemize}
\item {\textbf{$\vec{J}$} lies outside of all principal planes.}
%\item {$\mathcal{T}R_{y}(\pi)$ symmetry broken.}
\item {Two $\Delta\mathrm{I}$ = 1 bands of same parity with nearly same energy.}
\item {Opposite chirality bands.}
\end{itemize}
\vspace{-4pt}
\textbf{Standard fingerprints for chiral bands:}
\begin{itemize}
\item{Close excitation energies}
\item{Constant Staggering parameter}
\item{Identical B(M1)/B(E2) ratios}
\end{itemize}
\end{frame}


\begin{frame}{Triaxiality - Fingerprints (2/2)}
\vspace{-6pt}
\textbf{\large \color{red} Wobbling}
\vspace{-6pt}
\begin{center}
\begin{figure}
\includegraphics[width=0.28\textwidth]{wobb.png}
\hspace{15pt}
\includegraphics[width=0.35\textwidth]{wobb.pdf}
\end{figure}
\end{center}
\vspace{-7pt}
\begin{itemize}
\item {Analog of the spinning motion of an asymmetric top.}
\item {Oscillation of a principal axis about the space fixed \textbf{$\vec{J}$}.}
\end{itemize}
\vspace{-4pt}
\textbf{Standard fingerprints for Wobbling bands:}
\begin{itemize}
\item{Rotational bands corresponding to n$_{w}$ = 0, 1, 2, ....}
\item{Transitions from n$_{w+1} \rightarrow$ n$_{w}$ ($\Delta$n$_{w}$ = +1)}
\item{Interband Transitions are $\Delta\mathrm{I}$ = 1, E2} 
\end{itemize}
\end{frame}

%\begin{frame}{Signatures of Wobbling bands}
%\begin{itemize}
%\item{Rotational bands corresponding to n$_{w}$ = 0, 1, 2, ....}
%\item{Transitions from n$_{w+1} \rightarrow$ n$_{w}$ ($\Delta$n$_{w}$ = +1)}
%\item{Interband Transitions are $\Delta\mathrm{I}$ = 1, E2} 
%\end{itemize}
%\end{frame}

\begin{frame}{Previous cases of wobbling}{Discrepancies observed}
\begin{multicols}{2}
\begin{itemize}
\item{Until 2015, only five wobblers discovered:  $^{161}$Lu, $^{163}$Lu, $^{165}$Lu, $^{167}$Lu and $^{167}$Ta.}
\item{All in A$\sim$160 region.}
\item{Wobbling energy (E\textsubscript{wobb})- energy associated with wobbling excitations.}
\item{E\textsubscript{wobb} vs. spin - did not agree with the \textbf{Standard Wobbler} theory!}
\end{itemize}
\begin{figure}
\includegraphics[width=0.4\textwidth]{prev_wobb.png}
\end{figure}
\end{multicols}
\end{frame}

\begin{frame}{Simple wobbler}
\begin{columns}[c]
\begin{column}{0.4\textwidth}
\begin{figure}
\begin{center}
\includegraphics[width=\textwidth]{simple_wobb.png}
\end{center}
\end{figure}
\end{column}
\begin{column}{0.6\textwidth}
\begin{itemize}
\item{Wobbling motion in triaxial even-even nuclei.}
\item{Moment of Inertia (MOI) order: $\mathcal{J}_{m}>\mathcal{J}_{s}>\mathcal{J}_{l}$}
\item{Wobbling excitations - small amplitude oscillations of $\vec{J}$ about the m-axis.}
\end{itemize}
\end{column}
\end{columns}
\begin{align*}
E(n_{w},\mathnormal{J}) = \frac{\hbar^{2}}{2\mathcal{J}_{1}}J(J+1) + (n_{w} + \frac{1}{2})\hbar\omega\textsubscript{wobb}
\end{align*}
\begin{align*}
\space \space \space \space \hbar\omega\textsubscript{wobb} = \frac{J}{\mathcal{J}_{1}}\sqrt{\frac{(\mathcal{J}_{1}-\mathcal{J}_{2})(\mathcal{J}_{1}-\mathcal{J}_{3})}{\mathcal{J}_{2}\mathcal{J}_{3}}}
\end{align*}
\end{frame}

\begin{frame}{Quasiparticle Triaxial Rotor (QTR) model}
\begin{figure}
\begin{center}
\includegraphics[width=0.8\textwidth]{quasi.png}
\end{center}
\end{figure}
Triaxiality in odd-mass nuclei: High-j unpaired proton/neutron couples with the triaxial core $\implies$ Modified wobbling motion.
\end{frame}

\begin{frame}{Longitudinal Wobbler}{Types of Wobbling(1/2)}
\begin{multicols}{2}
\begin{figure}
\begin{center}
\includegraphics[width=0.35\textwidth]{long_wobb.png}
\end{center}
\end{figure}
\begin{itemize}
\item{Odd-particle aligned with axis with max. MOI (m-axis).}
\item{$\mathcal{J}_{3} > \mathcal{J}_{2}$ and $\mathcal{J}_{3} > \mathcal{J}_{1}$}
\item{Longitudinal wobbling: \\ $\mathcal{J}_{3}$ = $\mathcal{J}_{m}$}
\end{itemize}
\end{multicols}
\begin{center}
%\begin{align*}
%E\textsubscript{wobb}(\mathnormal{I}) = E(\mathnormal{I},n_{w}=1) - [E(\mathnormal{I}-1,n_{w}=0) + E(\mathnormal{I}+1,n_{w}=0)]/2
%\end{align*}
\begin{align*}
\hbar \omega_{w} = \frac{j}{\mathcal{J}_{3}} \left[\left(1+ \frac{J}{j} \left( \frac{\mathcal{J}_{3}}{\mathcal{J}_{1}} -1 \right) \right) \left( 1+ \frac{J}{j} \left( \frac{\mathcal{J}_{3}}{\mathcal{J}_{2}} -1 \right)\right)\right]^{1/2}
\end{align*}
\end{center}
$\implies$ E\textsubscript{wobb} increases with J.
\begin{center}
\Ovalbox{\small{S.Frauendorf and F.D\"{o}nau, Phys. Rev. C 89, 014322 (2014).}}
\end{center}
\end{frame}

\begin{frame}{Transverse Wobbler}{Types of Wobbling(2/2)}
\begin{multicols}{2}
\begin{figure}
\begin{center}
\includegraphics[width=0.35\textwidth]{trans_wobb.png}
\end{center}
\end{figure}
\begin{itemize}
\item{Odd-particle aligned perpendicular to axis with max. MOI (s- or l-axis).}
\item{$\mathcal{J}_{3} < \mathcal{J}_{2}$ and $\mathcal{J}_{3} > \mathcal{J}_{1}$}
\item{Transverse wobbling: \\ $\mathcal{J}_{3}$ = $\mathcal{J}_{s}$} 
\end{itemize}
\end{multicols}
%\begin{align*}
%E\textsubscript{wobb}(\mathnormal{I}) = E(\mathnormal{I},n_{w}=1) - [E(\mathnormal{I}-1,n_{w}=0) + E(\mathnormal{I}+1,n_{w}=0)]/2
%\end{align*}
\begin{align*}
\begin{centering}
\hbar \omega_{w} = \frac{j}{\mathcal{J}_{3}} \left[\left(1+ \frac{J}{j} \left( \frac{\mathcal{J}_{3}}{\mathcal{J}_{1}} -1 \right) \right) \left( 1+ \frac{J}{j} \left( \frac{\mathcal{J}_{3}}{\mathcal{J}_{2}} -1 \right)\right)\right]^{1/2}
\end{centering}
\end{align*}

$\implies$ E\textsubscript{wobb} decreases with J.
\begin{center}
\Ovalbox{\small{S.Frauendorf and F.D\"{o}nau, Phys. Rev. C 89, 014322 (2014).}}
\end{center}
\end{frame}

\begin{frame}{Longitudinal wobbler}
	\centering 
\animategraphics[height=3.1in, autoplay]{10}{images/long_wobb-}{0}{119}
\\\tiny{\textit{Courtesy Dr. J. T. Matta}}  
\end{frame} 

\begin{frame}{Transverse wobbler}
	\centering 
\animategraphics[height=3.1in, autoplay]{10}{images/trans_wobb-}{0}{119} 
\\\tiny{\textit{Courtesy Dr. J. T. Matta}}
\end{frame} 

\begin{frame}{Confirmation of Wobbling bands}
$\Delta\mathrm{I}$ = 1, E2 interband transitions between wobbling band are verified by:
\begin{itemize}
\item {\textbf{Angular Distributions}}
\end{itemize}
\begin{columns}[c]
\begin{column}{0.6\textwidth}
Observation of the intensity distribution of a single $\gamma$-ray transition relative to an orientation axis.
\begin{align*}
W(\theta) = A\textsubscript{0}(1 + A\textsubscript{2}P\textsubscript{2}(\cos\theta) + A\textsubscript{4}P\textsubscript{4}(\cos\theta))
\end{align*}
Mixing ratio: $\delta$ = $\sqrt{\frac{I_{\gamma}(L')}{I_{\gamma}(L)}}$
\end{column}
\begin{column}{0.4\textwidth}
\begin{figure}
\includegraphics[height=5cm,width=4.5cm]{ang_dis.png}
\end{figure}
\end{column}
\end{columns}
\end{frame}


\begin{frame}{Confirmation of Wobbling bands}
$\Delta\mathrm{I}$ = 1, E2 interband transitions between wobbling band are verified by:
\begin{itemize}
\item {\textbf{Directional Correlation of Oriented states (DCO) Ratios }}
\end{itemize}
\begin{columns}[c]
\begin{column}{0.6\textwidth}
\begin{align*}
R\textsubscript{DCO} = \frac{I_{\theta_{1}}^{\gamma_{2}}(\text{Gate}_{\theta_{2}}^{\gamma_{1}})}{I_{\theta_{2}}^{\gamma_{2}}(\text{Gate}_{\theta_{1}}^{\gamma_{1}})}
\end{align*}
\begin{itemize}
%\item {DCO-like ratio - more inclusive gates, higher intensity.}
\item {Calculated for weak transitions to establish their multipolarities.}
\item {Distinguishes between pure $\Delta\mathrm{I} = 1$ and pure $\Delta\mathrm{I} = 2$.}
\end{itemize}
\end{column}
\begin{column}{0.4\textwidth}
\begin{figure}
\includegraphics[height=5cm,width=4.5cm]{ang_dis.png}
\end{figure}
\end{column}
\end{columns}
\end{frame}


\section{Experiment}

\begin{frame}{Experiment}
\begin{center}
\textbf{Gammasphere at Argonne National Laboratory, USA}
\end{center}
\begin{multicols}{2}
\begin{figure}
\includegraphics[width=0.5\textwidth]{../gammasphere.jpg}
\end{figure}
\begin{itemize}
\item{110 Compton suppressed HpGe detectors arranged in spherical geometry.}
\item{Compton suppression achieved with large BGO scintillators.}
\item{Honeycomb suppression scheme.}
\item{Operated in Anti-coincidence mode.}
\item{Detectors arranged in 17 different angular rings around the beamline.}
\end{itemize}
\end{multicols}
\end{frame}

\begin{frame}{Experiment (cont..)}
\begin{columns}[c]
\begin{column}{0.5\textwidth}
\textbf{\center{Initial Observation}}
\begin{itemize}
\item{J.T. Matta et. al. (2015)}
\item{$^{135}$Pr identified as first case of wobbling in A$\sim$130 region.}
\item{Reaction Used: $^{123}$Sb($^{16}$O,4n)$^{135}$Pr. \\ E\textsubscript{beam} = 80 MeV}
\item{Indication of a second wobbling band but insufficient statistics.}
\end{itemize}
\end{column}

\begin{column}{0.5\textwidth}

\end{column}
\end{columns}
\end{frame}


\begin{frame}{Experiment (cont..)}
\begin{columns}[c]
\begin{column}{0.5\textwidth}
\textbf{\center{Initial Observation}}
\begin{itemize}
\item{J.T. Matta et. al. (2015)}
\item{$^{135}$Pr identified as first case of wobbling in A$\sim$130 region.}
\item{Reaction Used: $^{123}$Sb($^{16}$O,4n)$^{135}$Pr. \\ E\textsubscript{beam} = 80 MeV}
\item{Indication of a second wobbling band but insufficient statistics.}
\end{itemize}
\end{column}
\hspace{-30pt}
\vrule{}
\begin{column}{0.5\textwidth}
\textbf{\center{Second experiment}}
\begin{itemize}
\item{Performed in October, 2015.}
\item{Same reaction used.}
\item{Gammasphere updated to the digital system.}
\item{Data acquired in triple $\gamma$-coincidences.}
\item{4 times more data as compared to the previous experiment collected.}
\end{itemize}
\end{column}
\end{columns}
\end{frame}

\section{Data Analysis}

\begin{frame}{Gammasphere Calibration}
\begin{figure}
\includegraphics[width=0.73\textwidth]{eu_cal.pdf}
\end{figure}
\begin{multicols}{2}
\begin{figure}
\includegraphics[width=0.5\textwidth]{../../angdis_plots/Efficiency.pdf}
\end{figure}
\begin{small}
\begin{itemize}
\item{Data analyzed using \texttt{RADWARE} suite of codes.}
\item{Energy and efficiency calibration using standard $^{152}$Eu source.}
\item{Data sorted into $\gamma-\gamma$ matrices and $\gamma-\gamma-\gamma$ cubes.}
\end{itemize}
\end{small}
\end{multicols}
\end{frame}

\begin{frame}{$^{135}$Pr Level Scheme}
\centering
\small{What we know so far.. (\textit{J.T. Matta et. al., PRL 114, 082501 (2015))}}
\begin{figure}
\includegraphics[width=0.65\textwidth]{james_level_scheme.png}
\end{figure}
\end{frame}

\begin{frame}{$^{135}$Pr Level Scheme}
\centering
\small{What we know so far.. (\textit{J.T. Matta et. al., PRL 114, 082501 (2015))}}
\vspace{-10pt}
\begin{center}
\begin{tikzpicture}
\node at (6.5,1.20) {\color{red} \Large ($n_{w} = 1$)};
\node at (6.5, 0.45) {\color{red} \Large Wobbling};
\node at (6.5, 0.07) {\color{red} \Large Band};
\node [anchor=east] (1) at (2.78,1.35) {\Large 1};
\begin{scope}[xshift=1.5cm]
    \node[anchor=north, inner sep=0] (image) at (0.08,1.1) {\includegraphics[width=0.65\textwidth]{james_level_scheme.png}};
 \begin{scope}[x={(image.south)},y={(image.south)}]
         \draw [-stealth, line width=3pt, red] (1) -- ++(0.4,0.0);
    \end{scope}
\end{scope}
\end{tikzpicture}%
\end{center}
\end{frame}

\begin{frame}{$^{135}$Pr Level Scheme}
\centering
\small{What we know so far.. (\textit{J.T. Matta et. al., PRL 114, 082501 (2015))}}
\centering
\begin{tikzpicture}
\hspace{-5pt}
\node at (0.1,3.8) {\color{red} \small $\Delta\mathrm{I} = 2$ sequences};
\node at (0.1,3.3) {\color{red} \small of alternate};
\node at (0.1,2.8) {\color{red} \small signatures};
\node [anchor=north] (2) at (0.1,2.5) {\Large 2};
\begin{scope}[xshift=1.5cm]
    \node[anchor=south west,inner sep=0] (image) at (0,0) {\includegraphics[width=0.7\textwidth]{james_level_scheme.png}};
 \begin{scope}[x={(image.south)},y={(image.south)}]
          \draw [-stealth, line width=3pt, red] (2) -- ++(0.25,0.0);
    \end{scope}
\end{scope}
\end{tikzpicture}%
\end{frame}

\begin{frame}{$^{135}$Pr Level Scheme}
\centering
\small{What we know so far.. (\textit{J.T. Matta et. al., PRL 114, 082501 (2015))}}
\centering
\begin{tikzpicture}
\node [anchor=south] (3) at (2.8,5.5) {\Large 3};
%\vspace{1pt}
\node at (3.2,5.0) {\color{red} Strong $\Delta\mathrm{I} = 1$};
\node at (3.2,4.5) {\color{red} M1 transitions};
\begin{scope}[xshift=1.5cm]
    \node[anchor=south east,inner sep=0] (image) at (0,0) {\includegraphics[width=0.7\textwidth]{james_level_scheme.png}};
 \begin{scope}[x={(image.south)},y={(image.south)}]
          \draw [-stealth, line width=3pt, red] (3) -- ++(0.35,0.0);
    \end{scope}
\end{scope}
\end{tikzpicture}%
\end{frame}


\begin{frame}{$^{135}$Pr Level Scheme (extended)}
\begin{figure}
\includegraphics[width=0.65\textwidth]{yrast1.png}
\end{figure}
\end{frame}


%\begin{frame}{$^{135}$Pr Level Scheme (extended)}
%\centering
%\begin{tikzpicture}
%\node [anchor=north] (3) at (2.8,6.8) {\color{red} \Large 1};
%\begin{scope}[xshift=1.5cm]
%    \node[anchor=south west,inner sep=0] (image) at (0,0) {\includegraphics[width=0.6\textwidth]{135Pr_yrast.pdf}};
% \begin{scope}[x={(image.south)},y={(image.south)}]
%          \draw [-stealth, line width=3pt, cyan] (3) -- ++(0.3,0.0);
%    \end{scope}
%\end{scope}
%\end{tikzpicture}%
%\end{frame}


\begin{frame}{$^{135}$Pr Level Scheme (extended)}
\vspace{-10pt}
\centering
\begin{tikzpicture}
\node [anchor=north] (3) at (1.0,2.8) {\color{red} \Large 2};
\begin{scope}[xshift=1.5cm]
   \node[anchor=south west,inner sep=0] (image) at (0,0) {\includegraphics[width=0.8\textwidth]{sig.pdf}};
 \begin{scope}[x={(image.south)},y={(image.south)}]
          \draw [-stealth, line width=3pt, cyan] (3) -- ++(0.3,0.0);
    \end{scope}
\end{scope}
\end{tikzpicture}%
\end{frame}



\begin{frame}{$^{135}$Pr Level Scheme (extended)}
\vspace{-10pt}
\centering
\begin{tikzpicture}
\node [anchor=east] (2) at (4.80,0.6) {\color{red} \Large 3};
\begin{scope}[xshift=1.5cm]
    \node[anchor=north, inner sep=0] (image) at (0.08,1.1) {\includegraphics[width=0.85\textwidth]{dip2.pdf}};
 \begin{scope}[x={(image.south east)},y={(image.south east)}]
         \draw [-stealth, line width=3pt, cyan] (2) -- ++(0.1,0.0);
    \end{scope}
\end{scope}
\end{tikzpicture}%
\end{frame}



\begin{frame}{$^{135}$Pr Level Scheme (extended)}
\vspace{-15pt}
\begin{center}
\begin{tikzpicture}
\node at (6.50,0.85) {\color{red} \Large ($n_{w} = 2$)};
\node at (6.50,0.35) {\color{red} \Large Wobbling};
\node at (6.50,0.01) {\color{red} \Large Band};
\node [anchor=east] (1) at (5.00,0.01) {\color{red} \Large 4};
\begin{scope}[xshift=1.0cm]
    \node[anchor=north, inner sep=0] (image) at (0.05,1.3) {\includegraphics[width=10cm, height = 8.5cm]{2wobb.pdf}};
 \begin{scope}[x={(image.south)},y={(image.south)}]
         \draw [-stealth, line width=3pt, cyan] (1) -- ++(0.2,0.0);
    \end{scope}
\end{scope}
\end{tikzpicture}%
\end{center}
\end{frame}


\begin{frame}{$^{135}$Pr Level Scheme (all together)}
\centering
\vspace{-20pt}
\begin{figure}
\includegraphics[width=11cm, height=8cm]{135Pr_v.pdf}
\end{figure}
\end{frame}



\section{Preliminary Results}

\begin{frame}{Results (E\textsubscript{wobb} vs. Spin)}
\centering
\begin{align*}
E\textsubscript{wobb}(\mathnormal{I}) = E(\mathnormal{I},n_{w}=1) - [E(\mathnormal{I}-1,n_{w}=0) + E(\mathnormal{I}+1,n_{w}=0)]/2
\end{align*}
\begin{figure}
\includegraphics[width=0.6\textwidth]{../article/Ewobb.png}
\end{figure}
\textbf{\textit{Confirms Transverse Wobbling in $^{135}$Pr!}}
\end{frame}

\begin{frame}{Angular distributions (1/3)}
\centering
\textbf{n$_{w}$ = 1 $\rightarrow$ Yrast linking transitions}
\begin{columns}[c]
\begin{column}{0.5\textwidth}
\begin{figure}
\includegraphics[width=0.9\textwidth]{../747.pdf}
\\\includegraphics[width=0.9\textwidth]{../813.pdf}
\end{figure}
\end{column}
\begin{column}{0.5\textwidth}
\begin{figure}
\includegraphics[width=0.8\textwidth]{../754.pdf}
\\\includegraphics[width=0.4\textwidth]{connecting1.png}
\end{figure}
%\textbf{All n$_{w+1} \rightarrow$ n$_{w}$ linking transitions are $\Delta\mathrm{I}$ = 1,E2.}
\end{column}
\end{columns}
\end{frame}


\begin{frame}{Angular distributions (2/3)}
\centering
\textbf{n$_{w}$ = 2 $\rightarrow$ n$_{w}$ = 1 linking transitions}
\begin{columns}[c]
\begin{column}{0.5\textwidth}
\begin{figure}
\includegraphics[width=0.9\textwidth]{../450.pdf}\\
\includegraphics[width=0.9\textwidth]{../550.pdf}
\end{figure}
\end{column}
\begin{column}{0.5\textwidth}
\begin{figure}
\includegraphics[width=0.9\textwidth]{../517.pdf}\\
\includegraphics[width=0.6\textwidth]{connecting2.png}
\end{figure}
%\textbf{All n$_{w+1} \rightarrow$ n$_{w}$ linking transitions are $\Delta\mathrm{I}$ = 1,E2.}
\end{column}
\end{columns}
\end{frame}


\begin{frame}{Angular distributions (3/3)}
\centering
\textbf{Signature Partner $\rightarrow$ Yrast linking transitions}
\begin{columns}[c]
\begin{column}{0.5\textwidth}
\begin{figure}
\includegraphics[width=1.2\textwidth]{../594.pdf}
\end{figure}
\end{column}
\begin{column}{0.5\textwidth}
\begin{itemize}
\item {Exhibited a predominant \textbf{dipole} character.}
\item {Differentiating factor between proposed wobbling bands and the signature partner. }
\item {Higher linking transitions too weak to extract angular distributions.}
\end{itemize}
\end{column}
\end{columns}
\end{frame}


\begin{frame}{DCO-like ratios}
\vspace{-16pt}
\begin{figure}
\includegraphics[width=1.05\textwidth]{../../angdis_plots/dco.pdf}
\end{figure}
\end{frame}


\section{Conclusion and Future Work}

\begin{frame}{Conclusion and Future Work (1/3)}
\begin{itemize}
\item {Transverse wobbling has been investigated in the A$\sim$130 region.}
\item {The proposed two-phonon wobbling band has been identified.} 
\item {$\Delta\mathrm{I}$ = 1, E2 interband transitions between wobbling band verified by:}
\begin{itemize}
\item {Angular Distributions}
\item {DCO-like ratios}
\end{itemize}
\item {This is only the third such case after $^{163,165}$Lu.}
\end{itemize}
\end{frame}

\begin{frame}{Conclusion and Future Work (2/3)}
\begin{itemize}
\item {QTR model calculations to support the experimental findings need to be done.}
\item {Once confirmed, it would firmly establish wobbling in A$\sim$130 region.} 
\item {Possibility of the two dipole bands to be chiral partners of each other being explored.}
%\item {This would be the first ever case of a triaxial nucleus to exhibit both chirality and wobbling.}
\end{itemize}
\end{frame}


\begin{frame}{Conclusion and Future Work (3/3)}
\centering
\begin{tikzpicture}
\node [anchor=east] (2) at (7.00,0.4) {\color{red} \Large Chiral partners??};
\begin{scope}[xshift=1.5cm]
    \node[anchor=north, inner sep=0] (image) at (0.08,1.1) {\includegraphics[width=0.8\textwidth]{135Pr_v2.pdf}};
\node [anchor=south east] (3) at (1.80,0.1) {\Large };
 \begin{scope}[x={(image.south west)},y={(image.south)}]
         \draw [-stealth, line width=2pt, cyan] (3) -- ++(0.13,0.0);
    \end{scope}
\node [anchor=south west] (3) at (1.80,0.1) {\Large };
 \begin{scope}[x={(image.south east)},y={(image.south)}]
         \draw [-stealth, line width=2pt, cyan] (3) -- ++(0.13,0.0);
    \end{scope}
\end{scope}
\end{tikzpicture}
\end{frame}


\begin{frame}{Conclusion and Future Work (3/3)}
\textbf{Possible chirality of the two dipole bands\\}
\begin{columns}[c]
\begin{column}{0.5\textwidth}
\\\textit{\color{red} Remember?}\\
Standard fingerprints for chiral bands:
\begin{itemize}
\item{Close excitation energies}
\item{Constant Staggering parameter}
\item{Identical B(M1)/B(E2) ratios}
\end{itemize}
\end{column}
\begin{column}{0.5\textwidth}
\begin{figure}
\includegraphics[width=0.8\textwidth]{../../angdis_plots/dipoles/energy.pdf}\\
\includegraphics[width=0.8\textwidth]{../../angdis_plots/dipoles/stag.pdf}
\end{figure}

%\begin{figure}
%\includegraphics[width=0.5\textwidth]{../../angdis_plots/dipoles/be2.pdf}
%\end{figure}

\end{column}
\end{columns}
\textit{This would be the first ever case of a nucleus to exhibit both fingerprints of triaxiality - chirality and wobbling!}
\end{frame}


\begin{frame}
\begin{figure}
\includegraphics[width=0.6\textwidth,center]{ques.png}\\~\\
\includegraphics[width=0.3\textwidth,right]{ques2.png}
\end{figure}
\end{frame}


\begin{frame}
\centering
BACKUP
\end{frame}

\begin{frame}
\centering
\begin{figure}
\includegraphics[width=0.8\textwidth]{zoomed_conn.png}
\end{figure}
\end{frame}

\begin{frame}{Signature Partner Band}
\begin{itemize}
\item {
If rotation is about a principal axis of the nucleus - mean field invariant w.r.t $\mathcal{R}_{z}(\pi)$ 
\begin{align*}
\mathcal{R}_{z}(\pi)\ket{} = \exp^{-i\alpha\pi}\ket{}
\end{align*}

where $\alpha$ is the signature quantum number and leads to a selection rule for $\mathrm{I}$:

\begin{align*}
I = \alpha + 2n; n = 0, \pm 1, \pm 2, ....
\end{align*} 
}
\item {For odd-A nuclei, $\alpha$ takes values $\pm 1/2$. }
\item {A principal axis rotation leads to sequences of $\Delta\mathrm{I} = 2$ bands having alternate signatures.} 
%\item {The present work identifies transitions to differentiate between the signature partner and the wobbling bands.}
\end{itemize}
\end{frame}


\begin{frame}{Magnetic Rotation}
\begin{columns}[c]
\begin{column}{0.6\textwidth}
\begin{itemize}
\item {A sequence of strong $\Delta\mathrm{I}$ =1, M1 transitions.}
\item {j\textsubscript{p} and j\textsubscript{n} get lined up separately.}
\item {This gives rise to a large transverse magnetic dipole moment $\mu_{\perp}$ that rotates and gives rise to magnetic radiation.}
\item {\textbf{$\mathcal{R}_{z}(\psi)$ symmetry is broken - A rotational band is a consequence!}}
\end{itemize}
\Ovalbox{\tiny{S. Frauendorf, Reviews of Modern Physics, Volume 73 (2001)}}
\end{column}
\begin{column}{0.4\textwidth}
\begin{figure}
\includegraphics[width=\textwidth]{dipole.png}
\end{figure}
\end{column}
\end{columns}
\end{frame}

\begin{frame}{Polarization Measurements}
\begin{columns}[c]
\begin{column}{0.7\textwidth}
\begin{itemize}
\item {Distinguishes between the electric and magnetic nature of $\gamma$-rays.}
\item {Investigated from the observed asymmetry of the Compton scattering of $\gamma$-rays.}
\item {Clover detectors used to measure polarization.}
\end{itemize}
\end{column}
\begin{column}{0.3\textwidth}
\begin{figure}
\includegraphics[width=1.2\textwidth]{polarization.png}
\end{figure}
\end{column}
\end{columns}
\end{frame}

\begin{frame}{Polarization Measurements - Previous Results}
Performed at the Indian National Gamma Array (INGA) using Clover detectors.
\begin{figure}
\includegraphics[width=0.85\textwidth]{pol_asymmetry.png}
\end{figure}
\end{frame}	


\end{document}


